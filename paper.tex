\documentclass{article}
\usepackage{amsmath}
\usepackage{MnSymbol}%
\usepackage[utf8]{inputenc}

\title{On large cannonball numbers}
\author{Pazzaz}
\date{April 2019}

\begin{document}

\maketitle

If $s$ is the number of sides, then the $n$th $s$-gonal number is
$$
P(n, s) = \frac{n^2(s-2)-n(s-4)}{2}
$$
Consider then the sum
$$
S(k, s) = \sum_{n=1}^{k} P(n, s)=\frac{k(k+1)}{4}\Bigg(\frac{(s-2)(2k+1)}{3}-(s-4)\Bigg)
$$
The question that this paper investigates is when this sum is equal to an $s$-gonal number for the same $s$, or when
\begin{equation}
S(k, s) = \frac{n^2(s-2)-n(s-4)}{2}
\end{equation}
is true for some non-trivial values of n and k. Such values of $S(k, s)$ are called cannonball numbers.

Our main result is that (1) has solutions for all $s=3p+2, p>0$ and an expression that computes the explicit value of $k$ that solves the equation.
After computational tests, we conjectured that
$$
k= \frac{s^2-4s-2}{3}
$$

To prove our conjecture, we prove that the corresponding $n$ is an integer. First, we know that
\begin{equation}
n= \frac{\sqrt{8(s-2)P(n, s)+(s-4)^2}+(s-4)}{2(s-2)}
\end{equation}
We also have that
$$
S(\frac{s^2-4s-2}{3}, s)=\frac{(s^3-6s^2+3x+19)(s^2-4s-2)(s^2-4s+1)}{162}
$$
Let us now only consider the case $s=3p+2$
$$
S(\frac{s^2-4s-2}{3}, 3p+2)=\frac{\left(3p^3-3p+1\right)\left(3p^2-2\right)\left(3p^2-1\right)}{2}
$$
Now we can examine (2) but with $P(n, s)=S(\frac{s^2-4s-5}{3}+1, 3p+2)$.
\begin{equation*}
\begin{split}
n &= \frac{\sqrt{8(s-2)S(\frac{s^2-4s-2}{3}, 3p+2)+(s-4)^2}+(s-4)}{2(s-2)}\\&= \frac{\sqrt{324p^8-648p^6+108p^5+396p^4-108p^3-63p^2+12p+4}+3p-2}{6p}
\end{split}
\end{equation*}
Now we prove the numerator is divisible by $6p$.
\begin{equation*}
\begin{split}
&\sqrt{324p^8-648p^6+108p^5+396p^4-108p^3-63p^2+12p+4}+3p-2\\
&\equiv\sqrt{3p^2+4}+3p-2 \\
&\equiv\sqrt{9p^2-12p+4}+3p-2 \\
&\equiv\sqrt{(-3p+2)^2}+3p-2 \\
&\equiv-3p+2+3p-2\quad(mod\ 6p)\\
&=0
\end{split}
\end{equation*}
Therefore, the expression is always an integer, which means that n is always an integer. And so $S(\frac{s^2-4s-2}{3}, 3p+2)$ is always a cannonball number. $\blacksquare$
\end{document}

